\documentclass[a4paper,14pt, draft]{report}
\usepackage[T1,T2A,T2B,T2C]{fontenc}
\usepackage[utf8]{inputenc}
\usepackage[english, russian]{babel}
\usepackage{amsmath,amssymb,amsthm,amscd,amsfonts,mathrsfs,amsthm,mathtools}
\usepackage[14pt]{extsizes}
\usepackage{enumerate}
\usepackage{bussproofs}
\usepackage{etoolbox}
\usepackage{indentfirst}
\newtheorem{theorem}{Теорема}
\newtheorem{definition}{Определение}
\newtheorem{statement}{Утверждение}
\newtheorem{lemma}{Лемма}
\newtheorem{remark}{Замечание}
\newtheorem{example}{Пример}
\newtheorem{consequence}{Следствие}
\usepackage{soulutf8}
\usepackage{soul}
\usepackage{algpseudocode}
\def\proofname{\textsc{Доказательство}}
\linespread{1.25}
\usepackage{setspace}
\usepackage[left=30mm, top=15mm, right=10mm, bottom=20mm, nohead]{geometry}

\usepackage{todonotes}
\begin{document}
\begin{titlepage}
\begin{center}
\vspace{5em}

\textbf{\large Прикладная алгебра и теория чисел}
\end{center}
\end{titlepage}
\renewcommand\contentsname{Оглавление}
\setcounter{page}{2}
\tableofcontents\newpage
\chapter{Вводная информация}
Эта глава содержит определения, утверждения и теоремы о группах, кольцах и полях. Эта информация понадобится для понимания дальнейшего материала.
\section{Группы}
\begin{definition}[Группа]
\so{Группой $\mathfrak{G}$} называется четверка\\$(G, *^{(2)}, e^{(0)}, -1^{(1)})$, где
\[
    \begin{cases}
        x * (y * z) = (x * y) * z,\\
        x * e = x,\\
        x * x^{-1} = e
    \end{cases}
\]
\end{definition}
\begin{definition}[Абелева группа]
\so{Абелевой группой} называется группа, в которой $*$ коммутативна ($x * y = y * x$).
\end{definition}
\begin{definition}[Порядок]
\so{Порядок группы}, $ord~\mathfrak{G}$ --- количество элементов.
\end{definition}
\begin{definition}[Циклическая группа]
~\\$G = \{e, x^1, x^2, x^3, \ldots, x^{-1}, x^{-2}, x^{-3}, \ldots\}$
\end{definition}
\begin{definition}[Подгруппа]
$\mathfrak{G}$ - группа $(G, *, e, -1)$. И множество $H \subseteq G$. Тогда $\mathfrak{H}$ называется \so{подгруппой}, если замкнута относительно операций $*, e, -1$.
\end{definition}
\colorbox{red}{Продолжение следует...}
\section{Поля и кольца}
\colorbox{red}{Будет написано...}
\chapter{Помехоустойчивое кодирование}
\section{Метрика Хэмминга}
\colorbox{red}{Рисунок}
\begin{definition}[Метрика Хэмминга]
$\Sigma$ - алфавит, $n$ - длина слова. Слова $u,~v\in\Sigma^n$. Тогда \so{метрика Хэмминга}, $\rho(u, v)$ --- количество позиций в словах $u$, $v$, в которых они различаются.
\end{definition}
\begin{theorem}$\rho$ --- метрика.
\end{theorem}
\begin{proof}
Проверим все свойства метрик: \begin{itemize}
    \item $\rho(u, v) = 0 \Leftrightarrow u = v$
    \item $\rho(u, v) = \rho(v, u)$
    \item $\rho(u, v) \ge 0$
    \item $\rho(u, v) + \rho(v, w) = \rho(u, w)$
\end{itemize}

\colorbox{red}{Отрезки}
\end{proof}

$$\Sigma^m \xrightarrow{f}\Sigma^n\rightsquigarrow\Sigma^n\xrightarrow{g}\Sigma^m$$
$c$ --- кодовое слово\\$c'$ --- слово с ошибками

\colorbox{red}{Окружности}

\begin{theorem}Код обнаруживает $n$ ошибок $\Leftrightarrow$ $\rho(c_1, c_2) > n$ для любых кодовых слов $c_1$, $c_2$.
\end{theorem}
\begin{proof}$(\Rightarrow)$ Допустим $\rho(c_1, c_2)\le n$. $c_1$ и $c_2$ отличаются не более чем в $n$ позициях. Можно в $c_1$ сделать $n$ ошибок и получить $c_2$.

$(\Leftarrow)$ $\rho(c_1, c_2) > n$. Слово $c'$ содержит не больше $n$ ошибок, $c$ --- исходное слово. Следовательно, если $c \neq c'$ --- ошибки были.
\end{proof}
\begin{definition}[Наименьшее расстояние]
Наименьшее расстояние между кодовыми словами (\so{минимальное раастояние кода}) --- число измененных символов, необходимое для перехода одного кодового слова в другое.
\end{definition}

Минимальное расстояние кода является главной характеристикой кода.

\begin{theorem}Код может исправить $\le n$ ошибок $\Leftrightarrow$ минимальное расстояние этого кода $>2n$.
\end{theorem}
\begin{proof}$(\Rightarrow)$ Допустим, минимальное расстояние $\le 2n$. $$\rho(c_1, c_2)\le 2n$$

Существует $c'$: $\rho(c', c_1)\le n$ и $\rho(c', c_2)\le n$.\\$c'$ --- принятое сообщение. Исправление невозможно.

$(\Leftarrow)$ $\rho(c_1, c_2) > 2n$.

$c'$ --- слово с не более чем $n$ ошибками. Существует единственное кодовое слово $c$, для которого $\rho(c, c')\le n$. Следовательно, $c$ --- единствено возможный результат декодирования.
\end{proof}
\section{Примеры кодов}
\subsection{С проверкой на четность}
Алфавит $\Sigma = \{0, 1\}$, $m$ --- длина слов. Тогда $f$ --- кодирующая функция: $$f(u) = u \Big(\overset{m}{\underset{i=1}{\Sigma}} u\Big),$$ где $u\in\Sigma^m$.

Минимальное расстояние этого кода = 2. Следовательно, он может обнаружить 1 ошибку, но ни одной не может исправить.
\subsection{Дублирующий код}
Кодирующая функция $f$: $$f(u) = \underbrace{uu\ldots u}_\text{k раз},$$ где $u\in\Sigma^m$, $k\in\omega$.

Минимальное расстояние дублируещего кода равен количеству повторений ($k$). Следовательно, он может обнаружить $k-1$ ошибку, а исправить $\big[\frac{k-1}{2}\big]$. Основным минусом этого кода является то, что он порождает слишком длинные кодовые слова.
\section{Код Хэмминга}
$r\in\mathbb{Z^+}$. Числа $\neq0$ с двоичной записью длины $\le r$.

Матрица $r\times(2^r-1)$. Пусть $r = 3$, получается матрица $3\times8$:\[
\begin{pmatrix}
    0&0&1\\
    0&1&0\\
    0&1&1\\
    1&0&0\\
    1&0&1\\
    1&1&0\\
    1&1&1\\
\end{pmatrix}\]

$u$ --- исходное слово, $|u| = 2^r-1-r$. $v$ --- проверочная часть, $|v| = r$. Тогда кодовое слово - $uv$. Получаем ($2^r-1-r$, $2^r-1$)-код.

$v_i$: $i$-й столбец, просуммировать $u$ отмеченные 1. $$v_i = \underset{u}{\Sigma}~u_j\times a_{ij}$$
Минимальное расстояние: 3. \colorbox{red}{Добавить пояснение}

\colorbox{red}{Пример}

\section{Оптимальность}

$d > 2$

\colorbox{red}{Рисунок}

($m$, $n$)-код. $2^m (1+n) \le 2^n$
\begin{definition}[Совершенный код]
\so{Совершенный код} --- <<шары>> полностью закрывают пространство: $2^m (1+n) = 2^n$.
\end{definition}

Для кода Хэмминга: $2^{2^r-1-r}(1+2^r-1) = 2^{2^r-1}.$

\section{Групповые и линейные коды}
\begin{definition}
\so{Код групповой}, если множество кодовых слов --- аддитивная группа. \so{Код линейный}, если множество кодовых слов --- подпространство.
\end{definition}
\end{document}
