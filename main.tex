%\documentclass[14pt]{article}
\documentclass[a4paper,14pt, draft]{report}
\usepackage[T1,T2A,T2B,T2C]{fontenc}
\usepackage[utf8]{inputenc}
\usepackage[english, russian]{babel}
\usepackage{amsmath,amssymb,amsthm,amscd,amsfonts,mathrsfs,amsthm,mathtools}
\usepackage[14pt]{extsizes}
\usepackage{enumerate}
\usepackage{bussproofs}
\usepackage{etoolbox}
\usepackage{indentfirst}
\newtheorem{theorem}{Теорема}
\newtheorem{definition}{Определение}
\newtheorem{statement}{Утверждение}
\newtheorem{lemma}{Лемма}
\newtheorem{remark}{Замечание}
\newtheorem{example}{Пример}
\newtheorem{consequence}{Следствие}
\def\IF{\ifmmode\mathbf{if}\else\textbf{if}\fi}
\def\ENDIF{\ifmmode\mathbf{end~if}\else\textbf{end~if}\fi}
\def\THEN{\ifmmode\mathbf{then}\else\textbf{then}\fi}
\def\ELSE{\ifmmode\mathbf{else}\else\textbf{else}\fi}
\usepackage{soulutf8}
\usepackage{soul}
%\sodef\an{}{.4em}{1em plus1em}{2em plus.1em minus.1em}
\usepackage{algpseudocode}
%\renewenvironment{proof}{\textsc{Доказательство.}}{\qed}
\def\proofname{\textsc{Доказательство}}
%\patchcmd{\thebibliography}{\chapter*{\refname}}{}{}{}
\linespread{1.25}
\usepackage{setspace}
\usepackage[left=30mm, top=15mm, right=10mm, bottom=20mm, nohead]{geometry}

%\title{Coursework}
%\author{Vseslav Sekorin}
%\date{April 2016}

%
\usepackage{todonotes}
\begin{document}
%\maketitle
\begin{titlepage}
\begin{center}
\vspace{5em}

\textbf{\large Прикладная алгебра и теория чисел}
\end{center}
\end{titlepage}
\renewcommand\contentsname{Оглавление}
\setcounter{page}{2}
\tableofcontents\newpage
%\addcontentsline{toc}{chapter}{Вводная информация}
\chapter{Вводная информация}
Эта глава содержит определения, утверждения и теоремы о группах, кольцах и полях. Эта информация понадобится для понимания дальнейшего материала.
\section{Группы}
\begin{definition}[Группа]
\so{Группой $\mathfrak{G}$} называется четверка\\$(G, *^{(2)}, e^{(0)}, -1^{(1)})$, где
\[
    \begin{cases}
        x * (y * z) = (x * y) * z,\\
        x * e = x,\\
        x * x^{-1} = e
    \end{cases}
\]
\end{definition}
\begin{definition}[Абелева группа]
\so{Абелевой группой} называется группа, в которой $*$ коммутативна ($x * y = y * x$).
\end{definition}
\begin{definition}[Порядок]
\so{Порядок группы}, $ord~\mathfrak{G}$ --- количество элементов.
\end{definition}
\begin{definition}[Циклическая группа]
~\\$G = \{e, x^1, x^2, x^3, \ldots, x^{-1}, x^{-2}, x^{-3}, \ldots\}$
\end{definition}
\begin{definition}[Подгруппа]
$\mathfrak{G}$ - группа $(G, *, e, -1)$. И множество $H \subseteq G$. Тогда $\mathfrak{H}$ называется \so{подгруппой}, если замкнута относительно операций $*, e, -1$.
\end{definition}
\colorbox{red}{Продолжение следует...}
\section{Поля и кольца}
\colorbox{red}{Будет написано...}
\chapter{Помехоустойчивое кодирование}
\section{Метрика Хэмминга}
\colorbox{red}{Рисунок}
\begin{definition}[Метрика Хэмминга]
$\Sigma$ - алфавит, $n$ - длина слова. Слова $u,~v\in\Sigma^n$. Тогда \so{метрика Хэмминга}, $\rho(u, v)$ --- количество позиций в словах $u$, $v$, в которых они различаются.
\end{definition}
\begin{theorem}$\rho$ --- метрика.
\end{theorem}
\begin{proof}
Проверим все свойства метрик: \begin{itemize}
    \item $\rho(u, v) = 0 \Leftrightarrow u = v$
    \item $\rho(u, v) = \rho(v, u)$
    \item $\rho(u, v) \ge 0$
    \item $\rho(u, v) + \rho(v, w) = \rho(u, w)$
\end{itemize}

\colorbox{red}{Отрезки}
\end{proof}

$$\Sigma^m \xrightarrow{f}\Sigma^n\rightsquigarrow\Sigma^n\xrightarrow{g}\Sigma^m$$
$c$ --- кодовое слово\\$c'$ --- слово с ошибками

\colorbox{red}{Окружности}

\begin{theorem}Код обнаруживает $n$ ошибок $\Leftrightarrow$ $\rho(c_1, c_2) > n$ для любых кодовых слов $c_1$, $c_2$.
\end{theorem}
\begin{proof}$(\Rightarrow)$ Допустим $\rho(c_1, c_2)\le n$. $c_1$ и $c_2$ отличаются не более чем в $n$ позициях. Можно в $c_1$ сделать $n$ ошибок и получить $c_2$.

$(\Leftarrow)$ $\rho(c_1, c_2) > n$. Слово $c'$ содержит не больше $n$ ошибок, $c$ --- исходное слово. Следовательно, если $c \neq c'$ --- ошибки были.
\end{proof}
\begin{definition}[Наименьшее расстояние]
Наименьшее расстояние между кодовыми словами (\so{минимальное раастояние кода}) --- число измененных символов, необходимое для перехода одного кодового слова в другое.
\end{definition}

Минимальное расстояние кода является главной характеристикой кода.

\begin{theorem}Код может исправить $\le n$ ошибок $\Leftrightarrow$ минимальное расстояние этого кода $>2n$.
\end{theorem}
\begin{proof}$(\Rightarrow)$ Допустим, минимальное расстояние $\le 2n$. $$\rho(c_1, c_2)\le 2n$$

Существует $c'$: $\rho(c', c_1)\le n$ и $\rho(c', c_2)\le n$.\\$c'$ --- принятое сообщение. Исправление невозможно.

$(\Leftarrow)$ $\rho(c_1, c_2) > 2n$.

$c'$ --- слово с не более чем $n$ ошибками. Существует единственное кодовое слово $c$, для которого $\rho(c, c')\le n$. Следовательно, $c$ --- единствено возможный результат декодирования.
\end{proof}
%\section{Примеры кодов}
%\subsection{С проверкой на четность}
%\subsection{Дублирующий код}
%\section{Код Хэмминга}
\end{document}
